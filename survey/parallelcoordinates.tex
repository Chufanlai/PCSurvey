% !Mode:: "TeX:UTF-8"

%要运行该模板,LaTex需要安装CJK库以支持汉字.
%字体大小为12像素,文档类型为article
%如果你要写论文,就用report代替article
%所有LaTex文档开头必须使用这句话
\documentclass[12pt,twocolumn]{article}

%使用支持汉字的CJK包
\usepackage{CJKutf8}
\usepackage{indentfirst}
\usepackage[top=0.5in, bottom=0.5in, left=0.4in, right=0.4in]{geometry}
\usepackage[square,comma,numbers,sort&compress]{natbib}
\usepackage{graphicx}
\usepackage{booktabs}
\usepackage{tabularx}
\usepackage{enumitem}
\usepackage{titling}

\setlength{\parindent}{2em}
\setlength{\parskip}{0.5em}

\renewcommand{\contentsname}{目录}
\renewcommand{\listfigurename}{插图目录}
\renewcommand{\listtablename}{表格目录}
\renewcommand{\refname}{参考文献}
\renewcommand{\abstractname}{摘要}
\renewcommand{\indexname}{索引}
\renewcommand{\tablename}{表}
\renewcommand{\figurename}{图}

\setlength{\abovecaptionskip}{0em plus 0.3em minus 0.3em}
\setlength{\belowcaptionskip}{0em plus 0.3em minus 0.3em}
\setlength{\tabcolsep}{0.5em}
\setlength{\columnsep}{0.4in}

\linespread{1.1}

\setlist[1]{itemsep=0em,topsep=0em}

\renewcommand{\arraystretch}{1.3}

% make title left
\makeatletter
\renewcommand{\maketitle}{\bgroup\setlength{\parindent}{0pt}
\begin{flushleft}
  \LARGE {\textbf{\@title}}\vspace{1em}

  \normalsize{\@author}
\end{flushleft}\egroup
}
\makeatother

% change margin
\def\changemargin#1#2{\list{}{\rightmargin#2\leftmargin#1}\item[]}
\let\endchangemargin=\endlist

%开始CJK环境,只有在这句话之后,你才能使用汉字
%另外,如果在Linux下,请将文件的编码格式设置成GBK
%否则会显示乱码
\begin{CJK*}{UTF8}{gbsn}
\CJKindent
\CJKtilde

%这是文章的标题
%\title{动态图的可视分析方法综述}

%这是文章的作者
%\author{王祖超\hspace{1em}袁晓如}

%这是文章的时间
%如果没有这行将显示当前时间
%如果不想显示时间则使用 \date{}
\date{}

%以上部分叫做"导言区",下面才开始写正文
\begin{document}

\twocolumn[
  \begin{@twocolumnfalse}
    %这是文章的标题
    \title{平行坐标综述}
    %这是文章的作者
    \author{赖楚凡\hspace{1em}袁晓如}
    %先插入标题
    \maketitle
    %\vspace{-4em}

    \begin{changemargin}{0.2in}{0in}
    {\bf 摘\hspace{1em}要:}
    \vspace{1em}

    {\bf 关键词:}
    \vspace{1em}
    \end{changemargin}

    %这是文章的标题
    \title{Parallel Coordinates}
    %这是文章的作者
    \author{Chufan Lai\hspace{1em}Xiaoru Yuan}
    %先插入标题
    \maketitle
    %\vspace{-4em}

    \begin{changemargin}{0.2in}{0in}
    {\bf Abstract: } 
    \vspace{1em}

    {\bf Keywords: }
    \vspace{2em}
    \end{changemargin}
  \end{@twocolumnfalse}
]

\cite{inselberg1985plane}
%再插入目录
%\tableofcontents
%\section{引言}
%\label{section:introduction}
\bibliographystyle{abbrvnat}
%\setcitestyle{aysep={,},yysep={;}}
\bibliography{parallelcoordinates}
	
\end{CJK*}
\end{document}

